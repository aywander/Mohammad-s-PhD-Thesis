\section*{Abstract}
%A current important research topic in modern Astrophysics is to understand the formation process of galaxies like the Milky Way. Among the various components of the Milky Way Galaxy, the Stellar Halo preserves the most useful information about the evolution of the system. Despite a number of detailed studies over many years, the shape and extent of the Galactic Halo is still debated and knowledge about it is incomplete.
%
%The existence of the stellar streams in the halo provides important clues to the hierarchical formation of our galaxy. We study these substructures through the characteristics of various stellar populations. Among many, RR Lyraes are very good Galactic halo tracers as their relatively high intrinsic luminosity allows them to be detected out to large galactocentric radii. Also their well established absolute magnitudes permits accurate distance determinations. In addition, RRLs are metal poor and very old and can hold information of galaxy evolution. Their characteristic colors and light curves made them easy to be identified.
%
%Keller et al. (2008) found that the power-law slope of the RR Lyrae (RRL) space density distribution is steepened beyond the Galactocentric radius of $R\sim45$\,kpc. They identified 2016 RRL candidates derived from the analysis of archival observations of the Southern Edgeworth-Kuiper Belt Object (SEKBO) survey. We have investigated this result by following up on a subset of 137 candidates with a range of magnitudes ($V\sim14-20$) using the Faulkes Telescope (FT) database and confirmed 57 candidates as real RRLs. A cross-match between SEKBO RRL survey and Sloan Digital Sky Survey Data Release-7 (SDSS DR-7) revealed 272 RRL candidates in common. Applying the color selection criteria proposed by Ivezi{\'c} et al. (2005) resulted in 193 likely RRLs. The completeness as a function of magnitude was calculated empirically from the combined set of SEKBO RRL candidates from current FT data, SDSS cross-matched data, and the Prior et al. (2009) catalog. This resulted in a spatial density distribution characterized by two power laws with a break radius R within a range between 45\,kpc and 50\,kpc, similar to the results of Keller et al. (2008). We find the power-law slopes for the inner halo as $n_{\rm{inner}}=-2.78\pm0.02$ and for the outer halo as $n_{\rm{outer}}=-5.0\pm0.2$.
%
%The SkyMapper telescope survey promises us to deliver better quality data for the entire southern hemisphere sky. We quantify the likely efficiency and completeness of the survey as regards the detection of RR Lyrae variable stars. We have accomplished this via observations of the RR Lyrae-rich globular cluster NGC 3201. We find that for single epoch $uvgri$ observations followed by two further epochs of $gr$ imaging, as per the intended 3-epoch survey strategy, we recover known RRLs with an efficiency of $>90\%$. We also investigate boundaries in the gravity-sensitive single-epoch 2-colour diagram that yield high completeness and high efficiency (i.e. minimal contamination by non-RRLs).
%The availability of multi-epoch data is one of the big advantages of the SkyMapper survey over SDSS. The primary selection for RRLs in SkyMapper through variability and then location in the gravity sensitive diagram will help to find thousands of RR Lyares in the southern sky over the entire halo and will likely identify many new substructures. These will make a major contribution to the shape and the extent of the halo and will constrain formation models for the Galaxy.