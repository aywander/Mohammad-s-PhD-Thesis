\section*{Abstract}

%We present the first stage of a comprehensive investigation of the interactions of the jets in the radio galaxy Hydra A with the intracluster medium. 

An important research area of modern astronomy is to understand the physics of jets from Active Galactic Nuclei (AGN) and their interaction with the interstellar medium (ISM) and intracluster medium (ICM). Using a wealth of observational data, over the last few decades theoreticians have studied AGN jets both analytically and numerically. However, to date, a complete understanding on the jet energetics and composition, jet velocity, complex jet morphology and jet-ICM interaction is absent. This thesis aims to understand the energetics and composition of the jet near its origin, its interaction with the galaxy and cluster, focussing on detailed models of the inner structure of the Hydra A radio source. Analysing radio observations  of the inner lobes of Hydra~A by \citet{taylor90}, I confirm jet power estimates $\sim 10^{45} \rm \> ergs \> s^{-1}$ derived by \citet{wise07} from dynamical analysis of the X-ray cavities. With this result and a model for the galaxy halo, I explore the jet-intracluster medium interactions occurring on a scale of 30 kpc using relativistic hydrodynamic simulations. The key features of my modelling are that i) I identify the four bright knots in the northern jet at about 4, 7, 11 and 16 kpc (deprojected) from the radio core as biconical reconfinement shocks, which result when an over pressured jet starts to come into equilibrium with the galactic atmosphere ii) the curved morphology of the source and the turbulent transition of the jet to a plume are produced by the dynamical interaction of a precessing jet with the ICM.  

I study the inner 10~kpc of the northern jet by utilising two dimensional axisymmetric simulations. Through an extensive parameter space study I determine the position of the internal shocks in the jet as a function of the initial jet velocity and overpressure ratio. I also compare the oscillation of the jet boundary in my simulations with the observations. For a jet inclination $\theta = 42^\circ$ estimated by \citet{taylor93} from rotation measure asymmetry, I deduce that the jet velocity is approximately $ 0.8 \,c$ at a distance $0.5 \ \rm kpc$ from the black hole. The combined constraints of jet power, the observed jet radius profile along the jet, and the estimated jet pressure and jet velocity imply a value of the jet density parameter $\chi \approx 13$ for the northern jet.

To study the complex source morphology within 30~kpc (on the northern side) I generalise my axisymmetric model to a three dimensional jet-ICM interaction model incorporating jet precession. Utilising the jet parameters obtained from the best fit axisymmetric model, a wide range of precession periods and two values of the precession angle I produce a set of three dimensional models. With the precessing jet model I successfully reproduce key features of the inner 30~kpc of the Hydra A northern jet: i) Four bright knots along the jet axis at approximately correct locations ii) The curvature of the jet within 10~kpc iii) Turbulent transition of the jet to a plume iv) A misaligned bright knot in the turbulent flaring zone. The best matching model for the Hydra A northern jet gives a precession period $\sim1$~Myr and a precession angle $\sim20^{\circ}$. I estimate the Mach number $\sim$1.7 for the advancing forward shock associated with the plume, with an associated pressure jump $\sim2.7$ across it. This low Mach number and pressure jump indicates a gentle heating of the ICM by the source in its early stages.  



%An important research area of modern astronomy is to understand the physics of jets from Active Galactic Nuclei (AGN) and their interaction with the interstellar medium (ISM) and intracluster (ICM) medium. Using the wealth of observational data, over the last few decades AGN jets have been studied both analytically and numerically. However, to date a complete understanding on the jet energetics and composition, jet velocity and jet-ICM interaction is absent. This thesis aims to understand the energetics and composition of the jet near its origin, its interaction with the galaxy and cluster, and the inner structure of the Hydra A radio source. Analysing radio observations  of the inner lobes of Hydra~A by \citet{taylor90} I confirm the jet power estimates $\sim 10^{45} \rm \> ergs \> s^{-1}$ derived by \citet{wise07} from dynamical analysis of the X-ray cavities. With this result and a model for the galaxy halo, we explore the jet-intracluster medium interactions occurring on a scale of 10 kpc using two-dimensional, axisymmetric, relativistic hydrodynamic simulations. A key feature of my modelling is that I identify the three bright knots in the northern jet at about 4 and 7 and 12 kpc from the radio core as biconical reconfinement shocks, which result when an over pressured jet starts to come into equilibrium with the galactic atmosphere. Through an extensive parameter space study I determine the position of the internal shocks in the jet as a function of the initial jet velocity and overpressure ratio. I also compare the oscillation of the jet boundary in my simulations with the observations. For a jet inclination $\theta = 42^\circ$ estimated by \citet{taylor93} from rotation measure asymmetry, I deduce that the jet velocity is approximately $ 0.8 \,c$ at a distance $0.5 \ \rm kpc$ from the black hole. The combined constraints of jet power, the observed jet radius profile along the jet, and the estimated jet pressure and jet velocity imply a value of the jet density parameter $\chi \approx 13$ for the northern jet. I show that for a jet $\beta = 0.8$ and $\theta = 42^\circ$, an intrinsic asymmetry in the emissivity of the northern and southern jet is required for a consistent brightness ratio $\approx 7$ estimated from the 6cm VLA image of Hydra A. I also explore the possibility of different inclinations compatible with the observed flux density ratio and relativistic beaming. However, there are no satisfactory jet inclinations, which account for the knot spacing and which are consistent with a standard relativistic beaming explanation for the flux density ratio of the northern and southern jets. Different rest-frame emissivities may be caused by either different magnetic field strengths in the two jets or higher dissipation in the southern jet related to its larger number of knots and more twisted structure. 
