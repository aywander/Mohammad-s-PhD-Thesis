\chapter{Future Work}\label{future_work}
%\doublespacing
The study of Hydra A inner 10~kpc jet presented in this thesis uses an innovative method for the estimation of jet velocity by combining the information of the inner knot locations and the oscillation of the the jet boundary.

The detailed parameter study presented in this thesis using this technique provides best fit jet parameters, which can be used for a even more realistic three dimensional modelling of the dynamical interaction of the Hydra A jets and the cluster atmosphere.

In the following I describe the studies that could be done in future guided by the methodology I have used and the results we obtained in this thesis.

%\section{The turbulent transition of Hydra A jets}
%As I discussed in chapter~\ref{chap4} the turbulent transition of a jet is a three dimensional phenomenon, which cannot be captured well with the two dimensional axis-symmetric jet models. The turbulent transition of the jet originates from the shear layer at the jet boundary. This shear layer is Kelvin-Helmholtz unstable. The instability grows very rapidly and causes the jet to become turbulent in the region where the jet becomes subsonic due to the reconfinement shock deceleration. Beyond this point, and either side of the shear layer the flow is turbulent. In this context it is important to remember that two-dimensional turbulence behaves differently to three-dimensional turbulence: enstrophy (the square of the vorticity) is conserved in two-dimensional turbulence but not conserved in three-dimensional turbulence, and energy cascades in two-dimensional turbulence do not show the same scaling universality down to small scales as in three dimensions. The turbulent transition may therefore not be correctly captured in two-dimensional axis-symemtric simulations.
%
%One direct consequence of enstrophy conservation seen in my 2-D simulations is the perturbation of the jet boundary by the back-flowing plasma vortices that travel towards the computational symmetry axis. The ram pressure of the vortices pushes the jet boundary toward the symmetry axis and when the jet boundary layer reaches the symmetry axis the jet stream temporarily disappears until new jet plasma reaches the affected zones to reestablish the jet stream. This phenomenon is sometimes referred to as jet pinching and it affects the growth of instabilities in the shear layer. Jet pinching is an unavoidable numerical artifact associated with the axisymmetric jet model. Sufficiently resolved three dimensional jets are rarely pinched and their additional degree of freedom allows for axis-symmetry to be broken. Instead, a different physical instability is often observed, namely that of jet jittering. Jet jittering may itself increase the deceleration rate and, therefore, a more realistic three dimensional jet model is required to study the turbulent transition of the Hydra A jets.
% 
%In the case of the Hydra A northern jet, one may directly use the jet parameters obtained through the parameteric study to initialize models of three dimensional jet-ICM interaction and study the turbulent transitions of the jets.

\section{The large scale morphology of the Hydra A}
In this thesis I have modelled the inner 20~kpc structures of the Hydra A northern jet, including the width profile of the jet, the bright knots along its axis, the curvature of the jet and the jet to plume transition. This is a good starting point for a bottom up approach to study different scales of a very extended source like Hydra A. Following the evolution of the jet plasma, one could first investigate the inner 50~kpc radio plume, then the series of X-ray cavities, and finally attempt to explain the large scale structures in the X-ray images including the outer shock at approximately 200~kpc. Energy and mass transport measurements from three-dimensional simulations, in particular the transport of metal rich gas from the galaxy center, across these scales will provide answers to how cooling flows are suppressed, and allow direct comparisons with ICM temperature and metal distribution maps, and observations of cold filaments.

%In this thesis I have modelled the inner 20~kpc structures of the Hydra A northern jet, including the width profile of the jet, the bright knots along its axis, the curvature of the jet and the jet to plume transition. This is a good starting point for a bottom up approach to study different scales of a very extended source like Hydra A. Following the evolution of the jet plasma, one could first investigate the medium-scale structures such as the turbulent transition of the jet to a plume and the high dissipative zones described above, followed by the inner 50~kpc radio plume, then the series of X-ray cavities, and finally attempt to explain the large scale structures in the X-ray images including the outer shock at approximately 200~kpc. Energy and mass transport measurements from three-dimensional simulations, in particular the transport of metals from the galaxy center, across these scales will provide answers to how cooling flows are suppressed, and allow direct comparisons with ICM temperature and metal distribution maps, and observations of cold filaments.

\section{Magnetohydrodynamic models} 
In this thesis I use a pseudo synthetic emissivity as a function of pressure derived by \citep{sutherland07} to obtain the synthetic surface brightness image of the modelled jet. However, a proper calculation of synchrotron emission requires magnetic field. Therefore, a realistic study of the radio jets and the lobes requires magnetohydrodynamic (MHD) modelling of the source. One can incorporate magnetic filed into the purely hydrodynamic model I developed in this work and study the radio features in an even more realistic way.

%\section{A precessing jet model for Hydra A}
%In the models for the northern jet of Hydra A presented here, I used a straight jet assumption which is reasonable because the northern jet is only mildly bent. However, a more realistic three-dimensional model is also required in order to interpret the jet curvature. A precessing jet model could produce the observed bending. It is possible to model a three dimensional precessing jet with the parameters obtained in this thesis. Such a precessing jet model could be used to investigate the 10-30~kpc region of the northern plume region, for example, i) the bent jet structure, ii) two additional brighter knots in the northern jet beyond 10~kpc, and iii) the highly dissipative zone at 10-20~kpc. The precessing jet model is also important to study the impact of precession on the jet morphology on scales from 30 kpc to 200 kpc and the energy and material transport from the central black hole to the outer cluster atmosphere.

\section{The modelling of AGN jets displaying inner knots}
In this thesis, I showed that simultaneously modelling the data on the locations of the inner knots and the oscillation of the boundary of the Hydra A northern jet can be used to estimate the velocity of the jet. This method of estimation of the jet velocity can in principle be applied with modifications to other radio

\section{Modelling complex morphologies of other radio sources}
\begin{figure}
\centering
\includegraphics[width=\linewidth]{cenA.jpg}
\caption{ A comparison between the simulated source morphology (left panel) for model G ( precession period $P = 25$~Myr) and the radio morphology of Centurus A (inset of the right panel). Credit for the Centaurus A image: CSIRO.}
\label{f:cenA}
\end{figure}
\begin{figure}
\centering
\includegraphics[width=\linewidth]{m87.jpg}
\caption{ A comparison between the simulated source morphology (left panel) for model G ( precession period $P = 25$~Myr) and the radio morphology of m87 (inset of the right panel). Credit for the m87 image: NRAO.}
\label{f:m87}
\end{figure}
In my precessing jet models I notice a generic morphological feature of the precessing jet. In the first half cycle of the jet precession, we see a jet structure leading a balloon shaped lobe behind it. This type of jet-lobe morphology is common in some sources, for example, Centaurus A, M87 (inner lobe), J0116-473 etc.

 Fig.~\ref{f:cenA} and \ref{f:m87} shows the morphological similarities between the simulated radio image of the source (in the left panels) for model G (precession period $P = 25$~Myr) and the observed radio morphology of Centaurus A and M87 (in the right panels). The snapshots are captured when the jet reaches approximately half (Fig.~\ref{f:cenA}) and approximately quarter (Fig.~\ref{f:m87}) of its precession cycle. Therefore, modelling precessing jet can be a potential strategy to study sources with this type of jet-lobe morphology.  












