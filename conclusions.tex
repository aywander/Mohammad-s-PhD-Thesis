\chapter{Summary and Discussion}\label{conclusions}
%\doublespacing 

The main aim of this study has been to understand the physics of the inner jets in Hydra~A with a view to inferring parameters such as jet kinetic power, pressure, density and velocity, precession period, precession angle in large scale models of the radio source and its interaction with the cluster atmosphere. I have performed this study in two stages. First, I have studied the inner 10~kpc of the the northern jet, where the jet is nearly straight, using a two dimensional axisymmetric model. Second, I have generalised the axisymmetric model to a three dimensional precessing jet model in order to study the complex morphology of the northern part of the source within 20~kpc from the core. 

\section{Axisymmetric model}
I have focused on the following key features of the radio jet inside 10 kpc from the core i) the bright knots in the northern jet at $\sim 3.7 \rm \ , \ and 7.0$ kpc from the black hole and ii) the wave-like boundary of the northern jet. To this end, I have performed a series of two dimensional axisymmetric relativistic hydrodynamic simulations of the interaction of the northern Hydra A jet with the interstellar medium, particularly within the central $10 \> \rm kpc$. I have utilised the hydrodynamic code PLUTO \citep{mignone07} to perform the 2D axisymmetric models. 

To ensure that I used reasonable values for the jet parameters in my simulations, I have estimated the powers associated with the inner radio lobes of Hydra~A corresponding to inner X-ray cavities. I have used 4.6~GHz radio observations by \citet{taylor90} to estimate the inner cavity power and have compared them with the estimates of \citet{wise07} for the same cavities based on the X-ray data. I obtain powers for the northern and southern cavities $\approx 1.8\times 10^{44} \> \rm ergs \> s^{-1}$ and $2.0\times 10^{44} \> \rm ergs \> s^{-1}$ respectively. These estimates are consistent with the \citet{wise07} estimates $\sim 2 \times 10^{44} \ \rm ergs \ s^{-1}$ for both cavities.
Hence, I adopt the total jet power obtained by \citet{wise07} $P_{\rm jet}=10^{45}$ erg $s^{-1}$ from the summation of powers of all X-ray cavities as the value for the jet power in the numerical models. Other jet parameters, the jet pressure $p_{\rm jet}$ ($= 2 p_{\rm a} \rm \ and \ 5  \ p_{\rm a}$; $p_{\rm a} = $ ambient pressure) and the jet inlet radius $r_{\rm jet}$($=180, 150, \rm \ and \ 100~pc$) are chosen based on the 23~cm VLBA and 6~cm VLA data of Hydra A \citep{taylor90}.

On the basis of minimum pressure estimates I conclude that, in the lobes, $k$, the ratio of energy in other particles to that in relativistic electrons $\sim 10$. Moderate values of this parameter are supported by other recent studies: \citet{birzan08} estimated $k$ for a group of radio galaxies assuming that the radio lobes are in pressure equilibrium with the ambient medium. Their estimates include the Hydra A radio lobes at 1.4 GHz for which they obtained a value of $k \approx 13$. \citet{hardcastle10} studied the inverse-Compton X-ray emission from the outer Hydra A radio lobes and obtained values of $k \sim 17$ and 23 for minimum Lorentz factor cut-offs of $\gamma_1=1$ and 10 respectively. These estimates are all comparable given the different techniques used to derive them. 

For the X-ray atmosphere used in my simulations, I have constructed hydrostatic profiles for the Hydra A atmosphere by fitting and extrapolating the density and temperature data from the X-ray observations of \citet{david01}.

The results of my numerical models of the interaction of an initially conical and ballistic jet with the ambient medium support the idea that consecutive biconical shocks are responsible for the bright knots in the northern jets of Hydra A. With appropriate values of the initial jet pressure ratio and velocity the observed knot spacings and variation in jet radius are reproduced along a considerable section of the jet. 

%I did not model the Southern jet since it is more twisted and a straight jet model would be inappropriate; furthermore the radius of the southern jet as a function of distance from the core has not been observationally determined.

From the comprehensive parameter study in Chapter~\ref{chapter5} I have selected models Ciii, Civ and Cv as are the best fit models for the inner $\sim 10$ kpc radio structure of the northern jet. These jet models with initially conical and ballistic jet, which is over-pressured with respect to the environment by a factor of 5 produce four successive biconical reconfinement shocks before the jets become fully turbulent. The location of the first three shocks and the radius profile of the jet along the direction of its propagation closely match the location of the southern edge of the first three bright knots and the radius profile of the Hydra A northern jet. Constructing a synthetic surface brightness image I have shown that the biconical shocks produced in the simulated jet are associated with bright knots. For the best fit models of the northern jet, the jet parameters are the jet kinetic power $P_{\rm jet}=10^{45} \> \rm erg \rm s^{-1}$,  the jet inlet radius $r_{\rm jet}=100 \rm \> pc$, the jet over pressure ratio  = 5,  the jet density parameter $\chi = 20.41, 12.75, 7.24$ and the jet velocity (in units of the speed of light) $\beta = 0.75, 0.8 \rm \ and \ 0.85$. The estimated jet velocity for the northern jet of Hydra A $\approx 0.8 \rm \> c$ is consistent with recent observational and theoretical estimates of jet velocities in FRI jets determined by \citet{laing14}. In the course of modelling the surface brightness of 10 FRI radio sources they estimated a kpc scale jet velocity $\approx 0.8 \rm \> c$.

The brightnesses of the knots in the best fit model gradually increase with distance from the core, in a way that is qualitatively consistent with the observed jet. However the brightness ratio between the second and first knot and between the third and second knot for the simulated jet (run Ciii) $\approx 2.5$ and 1.14 respectively, differs from the observed brightness ratio $\sim8.7$ and $\sim3$. This discrepancy may arise as a result of the magnetic field increasing faster than the pressure along the jet and hence the assumption of $B^2/8\pi \propto p_{\rm jet}$ in the emissivity model would underestimate the emissivity increase along the jet. 

The inferred relativistic jet velocity $\approx 0.8$~c differs from the estimate based on the Doppler beaming $\approx 0.5$~c. Consequently I estimate a large flux density ratio, 33, of the approaching and receding jet compared to the observed value of 7. The additional parameter study in \S~\ref{s:b_r} shows that the combination $\beta = 0.5$, jet kinetic power $10^{45} \rm \ erg \ s^{-1}$ and an inclination angle $\theta=42^\circ$ is unable to produce the correct shock locations and the profile of the jet boundary for any feasible combination of the jet inlet radius and pressure. Hence, one possibility is to adopt $\beta= 0.8$ and to attribute the different flux density ratio to a difference in intrinsic rest-frame emissivities. For example, the flux density ratio may be overestimated in the best fit model since I assume that the magnetic field is the same in both jets. If I assume that the magnetic field is 2.5 times stronger in the southern jet, the flux density ratio would be 7. Another possibility is that the observed value of the flux density ratio is low since the southern jet is more dissipative as a result of its greater bending and the greater number of shocks. 

Another possibility for the discrepancy between estimated and measured flux density ratios is that the angle, $\theta$, between the jet and the line of sight, inferred from the rotation measure asymmetry \citep[see][]{taylor93} differs from $42^\circ$. This is certainly possible given the range $30^\circ \lesssim \theta \lesssim 60^\circ$  estimated by \citet{taylor93}. Hence, I have used the jet velocity as a parameter, calculated the inclination required to give a northern to southern flux ratio of 7, calculated the deprojected spacing between the first and second knots and compared this with the simulated spacing. The result of this comparison has been that the simulated and observed spacings do not agree except at the lowest possible jet velocities, consistent with a beaming interpretation, $\beta \approx 0.35$. I have argued that a solution for the jet velocity at around $\beta = 0.35$ is unappealing since it is unlikely that the optimal velocity for knot spacing would be fortuitously close to the lower limit from beaming. 
%Moreover, this would imply that the density parameter, $\chi \ga 300$, and this is inconsistent with an electron-positron jet for which there are good arguments based upon X-ray observations and modelling \citep{croston05a,croston14a}. My estimates of $\chi \sim 7 - 20$ for the models based upon an inclination of $42^\circ$ could also be criticised as being inconsistent with an electron-positron jet. However, values of this magnitude could be indicative of some entrainment at this distance form the core.

Taking into consideration the modelling of shock spacing, radius evolution of the jet, and surface brightness ratio, I conclude that the jet velocities $\gtrsim 0.8 \, c$ and that there is an intrinsic asymmetry between the rest-frame emissivities of the northern and southern jets. This may be a result of different magnetic fields (by about a factor of 2.5) or higher dissipation in the southern jet. 


The initial value (at 0.5~kpc) of the density parameter $\chi = \rho c^2 / 4 p$ derived from the simulations is also of interest for the parsec-scale value of this parameter. Assuming that the jet has constant velocity from the pc-scale outwards, $\rho \propto r_{\rm jet}^{-2}$ and $p \propto r_{\rm jet}^{-8/3}$ so that $\chi \propto r_{\rm jet}^{2/3}$. From the VLBI images of \citet{taylor96} $r_{\rm jet} \approx 1 \> \rm pc$ in the 15.4~GHz image. Hence the best fit value of $\chi = 12.75$ extrapolates to 0.59 -- consistent with an electron-positron jet with $\gamma_1 \sim 1$ or an electron proton jet with $\gamma_1 \sim 700$.

 
The conclusions of the axisymmetric models are subject to the assumption of a low magnetic pressure in the jet and I have provided some justification for this assumption, on the sub-parsec scale in \S~3 and the lack of magnetic collimation from the parsec to kiloparsec scale. Nevertheless, the magnetic field evolves along a jet, and its downstream strength and influence on the dynamics is an interesting issue. Moreover, the magnetic field is important for the calculation of synchrotron emission so that even if it passively transported, its evolution is important for the calculation of surface brightness. Hence, the inclusion of a magnetic field in future simulations is of interest. However, as \citet{spruit11a} has shown there is a lot more physics to consider in this case, in particular the modelling of reconnection of three-dimensional magnetic field. Thus, while magnetic effects are important to consider in future work, their consideration is well beyond the scope of this thesis, which I consider to be a useful first step in modelling features such as shock spacing and radial oscillations in order to estimate jet velocities.

\section{Precessing jet model}
With the axisymmetric model I have successfully reproduce the correct oscillations of the jet boundary and the first two bright knots inside 10~kpc of the Hydra A northern jet. In order to study features beyond 10~kpc, where the jet curves significantly, I have generalised the axisymmetric model to a three dimensional precessing jet model. The  three dimensional precessing jet model successfully reproduces the prominent features of the complex inner 20~kpc jet-lobe morphology in the northern side of Hydra A. 

%In Paper I we modelled the inner two bright knots of the Hydra A northern jet utilising axisymmetric straight jet simulations. In this paper we have developed that model by incorporating jet precession and studying the three dimensional interaction of the jet with the intracluster medium. Our three dimensional precessing jet model successfully reproduces the prominent features of the complex inner 20~kpc jet-lobe morphology in the northern side of Hydra A.  

I have performed a parameter space study using parameters obtained from the best fit axisymmetric model, a range of precession periods (1, 5, 10, 15, 20 and 25~Myr) and two precession angles (15$^{\circ}$ and 20$^{\circ}$). From the parameter study presented in chapter~\ref{chapter8} I find that model A with a precession period of 1~Myr and a precession angle of 20$^{\circ}$ produces the correct jet curvature, the correct number of knots, and the jet to plume transition at approximately the correct locations. Therefore I choose this model as the optimal model. 

The optimal model reproduces:
\begin{enumerate}
\item Four bright knots along the direction of the jet. The bright knots appear at the locations of the biconical shocks resulting from reconfinement shocks associated with recollimation of the jet by the ambient medium. The locations of the knots at approximately 4, 7, 10 and 14~kpc coincide reasonably well with the observed bright knots at approximately (3.7, 7.0, 11.0 and 16.0~kpc). 
\item The turbulent transition of the jet to a plume at approximately $9$~kpc compared to the observed transition location at 10~kpc. The initially supersonic jet is significantly decelerated by the first two reconfinement shocks and the transition to turbulence begins after the second knot.  
\item A turbulent flaring zone at approximately 10-20~kpc from the core. The back flowing jet plasma from the cocoon wall near the fourth knot produces strong turbulence in this region. The turbulence is responsible for the widening of the flow at approximately 10~kpc from the core. This simulated feature is consistent with the following observed feature of Hydra A. From the polarisation image \citep{taylor90} we see that the polarisation drops from $40\%$ in the collimated jet (until 10~kpc from the core) to $10\%$ in the flared region (10-20~kpc from the core) on the northern side of Hydra A. This drop in polarisation in this region is consistent with an increase in turbulence there.
\item A misaligned knot in the turbulent flaring zone. This feature is only produced in model A, supporting the choice of that model as the best match to Hydra A. 
\end{enumerate}

%suggests that the flaring zone (10-20~kpc from the core) could be turbulent. 

I have estimated the Mach number of the forward advancing shock $\approx 1.85$ from our optimal model. This low Mach number and the pressure jump ($\approx 3.4$) of the ambient medium associated with the forward shock  suggest a gentle heating of the of the ICM by the radio AGN in its initial phases of evolution as noted by \citet{mcnamara12}. 

Inclusion of magnetic fields in this study would be interesting, mainly for the production of more realistic synthetic surface brightness images. For instance, magnetic field amplification in the turbulent flaring region (10-20~kpc) of the northern jet may be a possible explanation for the increase in brightness there. However, a purely hydrodynamic model does not capture this effect. For this reason, in our optimal model, the ratio of the brightness between the initial jet (up to 10~kpc from the core) and the turbulent plume (10-20~kpc from the core) is not reproduced correctly. 

In the models presented here I have only considered the inner 20~kpc morphology of the northern jet. Since the initial jet radius (0.1~kpc) is very small compared to the extent of the inner lobe (50~kpc), modelling the entire inner lobe for a large range of parameters is unrealistic. However the parameters I have obtained by modelling the inner 20~kpc northern jet morphology can be used as input into future large scale studies of this source. 

On the southern side the initial 5~kpc trajectory of the jet is not well determined observationally. Therefore, adequate modelling of the southern side of the source as I have done for the northern side requires further high resolution observations of the source. 

Interpreting the realignment time-scale estimate by \citet{natarajan98} as the precession period of the jet, I have estimated the viscosity parameter of the accretion disk of Hydra A to be $0.03\le \alpha \le 0.15$. 
The lower values of $\alpha$ within this range are consistent with the prediction from quasi-steady MHD disk models \citep{parkin13b}. 